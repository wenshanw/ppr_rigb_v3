%%%%%%%%%%%%%%%%%%%%%%%%%%%%%%%%%%%%%%%%%%%%%%%%%%%%%%%%%%%%%%%%%%%%%%%%%%%%
% AGUJournalTemplate.tex: this template file is for articles formatted with LaTeX
%
% This file includes commands and instructions
% given in the order necessary to produce a final output that will
% satisfy AGU requirements, including customized APA reference formatting.
%
% You may copy this file and give it your
% article name, and enter your text.
%
%
% Step 1: Set the \documentclass
%
%

%% To submit your paper:
\documentclass[draft]{agujournal2019}
\usepackage{url} %this package should fix any errors with URLs in refs.
\usepackage{lineno}
\usepackage[inline]{trackchanges} %for better track changes. finalnew option will compile document with changes incorporated.
\usepackage{soul}
\linenumbers
%%%%%%%
% As of 2018 we recommend use of the TrackChanges package to mark revisions.
% The trackchanges package adds five new LaTeX commands:
%
%  \note[editor]{The note}
%  \annote[editor]{Text to annotate}{The note}
%  \add[editor]{Text to add}
%  \remove[editor]{Text to remove}
%  \change[editor]{Text to remove}{Text to add}
%
% complete documentation is here: http://trackchanges.sourceforge.net/
%%%%%%%

\draftfalse

%% Enter journal name below.
%% Choose from this list of Journals:
%
% JGR: Atmospheres
% JGR: Biogeosciences
% JGR: Earth Surface
% JGR: Oceans
% JGR: Planets
% JGR: Solid Earth
% JGR: Space Physics
% Global Biogeochemical Cycles
% Geophysical Research Letters
% Paleoceanography and Paleoclimatology
% Radio Science
% Reviews of Geophysics
% Tectonics
% Space Weather
% Water Resources Research
% Geochemistry, Geophysics, Geosystems
% Journal of Advances in Modeling Earth Systems (JAMES)
% Earth's Future
% Earth and Space Science
% Geohealth
%
% ie, \journalname{Water Resources Research}

\journalname{Geophysical Research Letters}


\begin{document}

%% ------------------------------------------------------------------------ %%
%  Title
%
% (A title should be specific, informative, and brief. Use
% abbreviations only if they are defined in the abstract. Titles that
% start with general keywords then specific terms are optimized in
% searches)
%
%% ------------------------------------------------------------------------ %%



\title{Greenland surface melt dominated by solar and sensible heating}

%% ------------------------------------------------------------------------ %%
%
%  AUTHORS AND AFFILIATIONS
%
%% ------------------------------------------------------------------------ %%

% Authors are individuals who have significantly contributed to the
% research and preparation of the article. Group authors are allowed, if
% each author in the group is separately identified in an appendix.)

% List authors by first name or initial followed by last name and
% separated by commas. Use \affil{} to number affiliations, and
% \thanks{} for author notes.
% Additional author notes should be indicated with \thanks{} (for
% example, for current addresses).

% Example: \authors{A. B. Author\affil{1}\thanks{Current address, Antartica}, B. C. Author\affil{2,3}, and D. E.
% Author\affil{3,4}\thanks{Also funded by Monsanto.}}

\authors{Wenshan Wang\affil{1}, Charles S. Zender\affil{1}, Dirk van As\affil{2}, Robert S. Fausto\affil{2}, Matthew K. Laffin\affil{1}}

% \affiliation{1}{First Affiliation}
% \affiliation{2}{Second Affiliation}
% \affiliation{3}{Third Affiliation}
% \affiliation{4}{Fourth Affiliation}

\affiliation{1}{Department of Earth System Science, University of California, Irvine, California, USA}%
\affiliation{2}{Geological Survey of Denmark and Greenland (GEUS), Copenhagen, Denmark}%
%(repeat as many times as is necessary)

%% Corresponding Author:
% Corresponding author mailing address and e-mail address:

% (include name and email addresses of the corresponding author.  More
% than one corresponding author is allowed in this LaTeX file and for
% publication; but only one corresponding author is allowed in our
% editorial system.)

% Example: \correspondingauthor{First and Last Name}{email@address.edu}

\correspondingauthor{Wenshan Wang}{wenshanw@uci.edu}

%% Keypoints, final entry on title page.

%  List up to three key points (at least one is required)
%  Key Points summarize the main points and conclusions of the article
%  Each must be 100 characters or less with no special characters or punctuation and must be complete sentences

% Example:
% \begin{keypoints}
% \item	List up to three key points (at least one is required)
% \item	Key Points summarize the main points and conclusions of the article
% \item	Each must be 100 characters or less with no special characters or punctuation and must be complete sentences
% \end{keypoints}

\begin{keypoints}
\item enter point 1 here
\item enter point 2 here
\item enter point 3 here
\end{keypoints}

%% ------------------------------------------------------------------------ %%
%
%  ABSTRACT and PLAIN LANGUAGE SUMMARY
%
% A good Abstract will begin with a short description of the problem
% being addressed, briefly describe the new data or analyses, then
% briefly states the main conclusion(s) and how they are supported and
% uncertainties.

% The Plain Language Summary should be written for a broad audience,
% including journalists and the science-interested public, that will not have 
% a background in your field.
%
% A Plain Language Summary is required in GRL, JGR: Planets, JGR: Biogeosciences,
% JGR: Oceans, G-Cubed, Reviews of Geophysics, and JAMES.
% see http://sharingscience.agu.org/creating-plain-language-summary/)
%
%% ------------------------------------------------------------------------ %%

%% \begin{abstract} starts the second page

\begin{abstract}
The Greenland Ice Sheet is 
the primary source of global barystatic sea-level rise, 
and at least half of its total current mass
loss is attributed to surface melt. Here we use in-situ measurements
from 23 automatic weather stations on the ice sheet to identify the
dominant processes that drive all melt. Attention generally focuses on
large melt events impacting Greenland ice sheet mass loss, yet we show
that these anomalous events contribute only 2\% to
total surface mass loss. ~After filtering out the seasonal and diurnal
cycles of surface melt driven by net shortwave radiation, we find that
melt variability is 30\% driven by sensible heat exchange, 27\% by
shortwave radiation, 14\% by latent heat, and 12\% by~longwave
radiation, the latter two of which have been deemed more influential in
previous studies. Sensible heat exchange and shortwave radiation
correlate with the occurrence of dry and fast downslope winds. These
katabatic winds enhance vertical mixing that reduces the
temperature inversion and increases sensible heating of the surface. The
concomitant low humidity and clear skies are associated with increased
solar heating. Although katabatic winds are stronger in winter,
wind-driven surface melt dominates the Greenland ice sheet surface melt
on synoptic timescales in summer.
\end{abstract}

\section*{Plain Language Summary}
[ enter your Plain Language Summary here or delete this section]


\section{Introduction}
Surface melt and subsequent runoff, 
responsible for about 60\% of the total mass loss of 
the Greenland Ice Sheet \cite{VanDenBroeke2016}, 
have accelerated since the mid-1990s \cite{Fettweis2017}, 
making Greenland the primary cryospheric source 
of sea-level rise \cite{VanDenBroeke2016}, 
and a possible cause for the slowdown of 
the thermohaline circulation in the North Atlantic Ocean \cite{Rahmstorf2015}.
Massive melt events, 
such as two ice-sheet-wide melt episodes in July 2012, 
drew considerable attention from 
the public and scientific communities \cite{Bennartz2013, Tedesco2013, Hanna2014, Neff2014, Fausto2016a}. 
However, melt during these two episodes only contributed 12-15\% of 
total surface melt in 2012 \cite{Fausto2016a}. 
To investigate the origins of Greenland ice sheet surface melt 
throughout the entire melt season, 
we use in-situ measurements of surface ablation, radiation, 
and meteorology from 23 automatic weather stations (AWS) 
in the ablation zone \cite{VanAs2011b} (Fig. S1 and Table S1).

Net shortwave and longwave radiation largely 
determine the seasonal and inter-annual variability 
of surface melt 
\cite{VandeWal2005, VandenBroeke2008a, KuipersMunneke2018}.
Shortwave radiation provides the largest surface energy flux 
in the ablation zone \cite{VandenBroeke2008a}. 
Part of the acceleration of surface melt from 1995-2009 
is attributed to enhanced solar heating caused 
by reduced mid to high-level cloud cover 
\cite{Hofer2017, Ding2017}. 
On the other hand, longwave heating associated with 
increasing low-level clouds and moisture levels 
might be responsible for increased surface melt 
at Summit and along the western slope of 
the Greenland ice sheet 
\cite{Bennartz2013, Miller2015, VanTricht2016, Mattingly2018}. 

Non-radiative energy fluxes (such as turbulent heat fluxes) 
also play a crucial role in surface melt 
in the low-elevation ablation zone \cite{VandenBroeke2011, KuipersMunneke2018a} 
and during massive melt events \cite{Fausto2016a}. 
In-situ measurements reveal that 
non-radiative energy fluxes dominated melt variability 
along the western slopes of the Greenland ice sheet 
during the wide-spread advection-forced melt episodes 
in July 2012 \cite{Fausto2016a}. 
In the lower ablation zone, 
advection of relatively warm air over the nearby tundra 
generally dictates surface melting 
due to increased turbulent heat exchange 
in summer \cite{VandenBroeke2011, KuipersMunneke2018a}. 
Katabatic winds, which predominate over the Greenland ice sheet 
in both winter and summer, 
also enhances turbulent heat exchange 
by enhancing vertical mixing 
in the stable Arctic boundary layer \cite{VandenBroeke2009a}. 

\section{Data and Methods}
\subsection{Automatic Weather Stations}
We use AWS from 
the Programme for Monitoring of the Greenland Ice Sheet (PROMICE) \cite{Fausto2019}, 
which provides hourly measurements of meteorology, ablation, 
and four-component radiation. 
The network established in 2007. 
By 2018, it has operated 26 stations ($\sim 228$ station-years of raw data) 
in all eight drainage basins in the ablation zone of Greenland. 
The stations are usually arranged in pairs with 
one station close to the equilibrium line 
(station names suffixed with U for upper) 
and the other deeper into the ablation zone 
(station names suffixed with L for lower). 
In this study, we use 23 stations 
that are on ice and have at least two complete melt seasons of 
valid measurements (a total of $\sim 215$ station-years; Table S1). 
The meteorological and radiative data have gone through 
rigorous data quality control using the same methods \cite{Wang2016, Wang2018, Wang2018a}. 

\subsection{Surface Ablation Rate}
Surface ablation rate is estimated using 
surface height changes measured 
by sonic rangers and pressure transducer assembly (PTA) \cite{Fausto2012a}.
Every PROMICE AWS is equipped with two sonic rangers 
and one PTA to measure the ablation and/or accumulation of 
snow, ice, and both. 
The sonic ranger on the main station stake that 
stands on ice surface sinks as ice melts, 
and therefore only measures the snow height changes 
(i.e., snow ablation and accumulation). 
The sonic ranger on a separate stake 
drilled into ice measures surface height changes caused by 
both snow and ice. 
A PTA consists of a pressure transducer drilled into 
ice and a bladder above ground connected through 
a hose filled with antifreeze liquid. 
It converts the pressure changes into surface ice ablation 
(not recording ice accumulation). 
We use surface ablation rates estimated from 
these three instruments to cross-validate and fill missing values. 
Although the manufacturing uncertainty of sonic rangers 
($\sim 0.010\,m$) is smaller than that of a PTA 
($\sim 0.025\,m$), 
the former degrades in the field 
due to repeating cycles of riming \cite{Fausto2012a}. 
Therefore, when ice ablation rates estimated from 
sonic rangers and PTA conflict, 
we trust PTA, 
unless albedo and net energy fluxes suggest otherwise. 
To obtain a robust value, 
we estimate the net ablation in the past 24 hours at every hour. 
To remove the long-spike noises from 
sonic ranger measurements, 
we calculate a 24-hr moving median of the 24-hr net ablation rate. 
To keep a consistent timescale, 
we apply a 24-hr moving average on 
all other measurements except wind direction, 
which filters out diurnal cycles. 
We define the onset (refreeze) date of melt seasons 
as the first (last) day of 168 consecutive hours (7 days) 
with 24-hr ablation larger than $0.01\, m$
and air temperature higher than $-5\mathrm{^{\circ}C}$
(air temperature is usually higher than surface temperature). 
In this way, we do not include short-time ablation 
beyond these dates since this study 
focuses on common melt rather than extreme melt events. 

\subsection{Sensible and Latent Heat Fluxes}
Sensible and latent heat fluxes 
from PROMICE are estimated using the bulk method, 
the Monin-Obukhov similarity theory (MOST) \cite{VanAs2005, VanAs2011a}. 
The roughness lengths are assumed to be constant 
for snow ($5 \times 10^{-4} \, m$) 
and ice ($5 \times 10^{-3} \, m$) \cite{Fausto2016a} 
since the roughness elements of the snow/ice surface 
do not change significantly \cite{VanAs2005}. 
Nevertheless, assumptions in the bulk method 
introduce considerable uncertainties in the magnitude of 
estimated sensible and latent heat fluxes \cite{VandenBroeke2009a}. 
This is another reason we use correlation rather than 
magnitude to quantify contributions of 
component energy fluxes to surface ablation. 
Although estimated using temperature and wind speed, 
sensible heat has a better correlation with 
measured surface ablation than either of these two. 

\subsection{Moving correlation}
To attribute melt-causation to a specific component energy flux, 
we use 168-hr (7-day) moving correlation 
instead of one correlation coefficient throughout 
the whole melt season 
since the dominant energy flux could alternate between 
shortwave radiation, longwave radiation, sensible heat, 
and latent heat on synoptic timescales ($\sim 7$ days) 
depending on the leading physical processes. 
We test window lengths from 120 hours (5 days) to 264 hours (11 days). 
%- ww: need to change here --> no mentions of conclusion
All lengths lead to the same conclusion 
that shortwave radiation and sensible heat fluxes 
dominate the surface ablation on sub-seasonal timescales 
while a 7-day window yields the highest number of 
significant correlation coefficients. 


\section{In-situ Measurements of Surface Melt}
While satellites can monitor the surface melt extent of 
the Greenland ice sheet with broad spatial coverage, 
in-situ measurements are the only way to verify the surface mass balance.
Previous studies using in-situ measurements 
often focus on regions of Greenland \cite{VanAs2011a, VandeWal2005, VandenBroeke2011}
or specific years \cite{Fausto2016a}. 
In this study, we estimate in-situ surface mass loss 
in eight ablation regions 
around the entire Greenland Ice Sheet, 
with decade-long records at some sites 
(Fig. 1a and snow-only ablation in Fig. S2). 
Similar to the spatial distribution of 
surface temperature and melt frequency \cite{Mernild2011, Hall2013}, 
annual surface ablation increases towards the south and west, 
and lower elevations of the Greenland ice sheet. 
The latitudinal gradient of annual net ablation 
based on these stations is $0.32 \pm 0.21\,m$ 
per year per degree of latitude (in water equivalent, w.e.) 
along the western ice sheet margin, and $0.10 \pm 0.06\,m\,w.e.\,yr^{-1}\,degree^{-1}$ along the eastern margin, 
using the lowest-elevation stations. 
The downslope net ablation gradient 
is $4.72 \pm 2.75 \,m\,w.e.\, yr^{-1}\, km^{-1}$ 
in the west and $3.40 \pm 1.62 \,m\,w.e.\, yr^{-1}\, km^{-1}$ in the east.
Contrary to the case for melt extent \cite{Nghiem2012}, 
the year 2012 was not record-setting at all stations. 
Instead, for stations south of KAN ($\sim 67\mathrm{^{\circ} N}$), 
2010 is the year with the largest net ice ablation. 
The massive melt events in terms of extent 
are related to synoptic episodes in summer, 
whereas net ablation better reflects 
the ice-atmosphere energy exchange over all seasons \cite{Valisuo2018}. 

Although massive melt events abruptly
alter surface conditions and generate large surface mass loss 
in a short amount of time \cite{Fausto2016a}, 
they do not occur very often \cite{Nghiem2012}.
To compare contributions from melt events with different melt rates, 
we partition the total surface mass loss by 
deviations in daily melt rate from the mean daily melt rate 
for each station on that day of the year (Fig.~1b). 
Days with ablation rates within one standard deviation of 
the mean daily ablation rate contribute roughly 70\% to annual net ablation.
Contributions from large melt events exceeding 
three standard deviations above the average, 
such as the two in July 2012, are on average minimal (2\%). 
To understand the causes for the majority of surface mass loss, 
we must examine common, high-frequency, low melt-rate events.

Surface ablation has a clear seasonal cycle peaking in July, 
as a result of the seasonal cycle in 
the surface energy balance \cite{VandenBroeke2011}. 
To illustrate the seasonal ablation cycles at the stations, 
we calculate 31-day moving average ablation 
in different latitudinal zones (Fig.~1c). 
From north to south, the maximum daily melt rate increases 
and the melt season duration elongates. 
This seasonal cycle is determined mainly 
by shortwave radiation through albedo and solar zenith angle (Fig.~S3). 
The significant ($p<0.01$) correlation coefficient of 
daily ablation with net shortwave radiation is 0.75, 
while it is 0.30 with longwave radiation, 
0.52 with sensible heat, and 0.50 with latent heat. 
After filtering out the seasonal cycle in net shortwave radiation, 
only day-to-day variability remains. 
The rest of this study focuses on the energy budgets 
and leading processes of this sub-seasonal melt variability. 

\section{Sensible and Solar Heating Dominate Sub-Seasonal Ablation Variability}
%- ww: capitalize Sub-Seasonal?
Surface mass loss is largely controlled by the surface energy budget. 
To pinpoint the leading physical processes responsible 
for sub-seasonal surface ablation, 
we first identify the dominant coeval energy fluxes.
Surface ablation varies greatly on the sub-seasonal timescale (Fig.~1c).
Estimates of melt contribution from 
component energy fluxes based solely on flux magnitudes 
will misrepresent this variability.
Therefore, we attribute melt-causation 
to a specific component energy flux 
only when that flux is positive and temporally correlated 
($r > 0.5, p < 0.05$) with melt; 
otherwise, no causation is assumed and the attributed fraction is zero. 
We then partition surface ablation into fractions 
attributed to each sub-seasonal energy flux (Fig.~2). 
Averaged over all stations, sensible heat exchange (30\%) 
and net shortwave radiation (27\%) 
dominate the sub-seasonal surface ablation at all stations and for all years.
Net longwave radiation (14\%) and latent heat (12\%) are less important. 
This dominance of sensible and solar heating is 
due to a better temporal correlation rather than a higher magnitude. 
The fraction of time that melt is temporally correlated with 
sensible heat (50\%) and shortwave radiation (39\%) 
exceed fractions for latent heat (28\%) and longwave radiation 
(27\%; Note that the total fraction can exceed 100\% 
because melt can simultaneously correlate with multiple energy fluxes).
Shortwave radiation explains more melt than sensible heat 
in southern Greenland due to the insolation 
increase from the reduced solar zenith angle. 
Surface albedo is not significantly (anti-)correlated 
with melt on sub-seasonal timescales. 
The high correlation of ablation with shortwave radiation 
mainly stems from the surface insolation.

Melt during days when none of the above four energy fluxes 
are attributable is labeled as "None" (Fig.~2), 
and may be due to our methodology and/or neglected energy sources, 
i.e. rain and sub-surface energy fluxes.
The "None" ratio rises towards the lower ablation zone in the south,
which could be associated with rain energy fluxes.
Liquid precipitation in the warmer regions adds to the surface energy balance, 
yielding surface melt \cite{Fausto2016a}, 
though without measuring the quantity and temperature of rain, 
the energy contribution cannot be calculated. 
We also exclude any sub-surface energy flux 
since it is not measured at the AWS; 
its contribution to surface melt is small at $\sim 2-3 \, W/m^{2}$
as measured in western Greenland \cite{VandenBroeke2011, VanAs2011a}.


\section{Sensible and Solar Heating Enhanced During Katabatic Winds}
Surface energy budgets are affected by several physical processes 
ranging in size from synoptic-scale (advection) \cite{Hanna2013}
to regional scale (katabatic winds) \cite{VanDenBroeke1996} and 
to local-scale (barrier winds) \cite{VanDenBroeke1996}. 
Wind direction can help reveal the dominant wind forcings. 
Warm advection usually originates from the south; 
katabatic (i.e., gravity-driven) winds are mostly downslope; 
barrier winds are along the ice sheet margin. 
The Greenland ice sheet's topography,
a fairly flat interior and steeper slopes along the periphery,
helps generate katabatic and barrier winds.
The persistent anticyclonic conditions around Greenland 
since the 1990s \cite{Hanna2018} enhance southerly advection 
in the west of Greenland \cite{Hanna2013} 
as well as katabatic winds along the coasts \cite{Turton2019}.
To identify the dominant physical process, 
we estimate wind frequency, melt rate, and energy fluxes 
at different wind-from directions (Fig.~3). 
During the melt season, downslope winds prevail at the AWS sites. 
Although katabatic winds are stronger and 
more frequent in winter \cite{Gorter2014}, 
they can prevail in summer as well \cite{Vihma2011}. 
The two stations not dominated by downslope winds are MIT and NUK\_K, 
both of which are located on small independent mountain glaciers 
where the small length scales 
do not favor generating persistent katabatic winds. 
The average melt rate (and accumulative melt total) is larger 
during predominantly downslope wind than for other wind directions, 
and so are the average sensible heat flux 
and net shortwave radiation for all AWS sites. 
Net longwave radiation, on the other hand, 
increases when winds blow from the south or the ocean. 
This result indicates that downslope winds 
enhance sub-seasonal surface melt through sensible and solar heating.

The katabatic origin of these downslope winds is evident 
from the alignment of melt rates, component fluxes, 
and environmental conditions relative to 
the dominant wind direction at each station (Fig.~4a and b).
The dominant winds are fastest and driest (Fig.~4b), 
and produce the most melt when sensible and solar heating maximize (Fig.~4a).
Faster winds enhance vertical mixing to draw down the warmer air from above 
and increase sensible heating 
\cite{Parish1989, VandenBroeke2009a, KuipersMunneke2018} (Fig.~4a). 
Drier air and clear-sky conditions 
during katabatic winds transmit more insolation to heat the surface (Fig.~4a).
With more sensible and solar heating, 
surface melt accelerates during katabatic winds (Fig.~4a). 


\section{Discussion}
Decadal coeval records of melt rate and surface energy components 
at $~23$ AWS in Greenland's ablation zone reveal that 
katabatic winds dominate sub-seasonal ablation variability. 
Previous studies show significant contributions to 
Greenland's surface melt from warm southerly advection
\cite{Hanna2013, Neff2014, Mattingly2018, Oltmanns2019}.
Dry advection enhances surface melt through turbulent heat fluxes 
while moist advection heats the surface more through 
longwave radiation \cite{Tjernstrom2019}.
This warm advection, also favored by the persistent 
anti-cyclonic condition over Greenland, 
correlates well with surface warming 
in western Greenland over the past 20 years \cite{Hanna2013}. 
From 2000 until the early 2010s, 
abnormally active atmospheric rivers transported heat and moisture to 
and increased melt in this same area \cite{Mattingly2018}. 
On the other hand, frequent cyclones transported more moisture 
to the southeastern coast of Greenland 
during the same time periods \cite{Edwards-Opperman2018, Oltmanns2019}. 
This warm and possibly moist advection helped 
cause the record surface melt extent in 2012
\cite{Tedesco2013, Hanna2014, Neff2014}.
However, these large spatial-scale studies focus on 
melt extent observed by satellites or 
simulated by models rather than on in-situ measured ablation. 
The former favors synoptic-scale over regional/local-scale influences. 
Our analysis using in-situ measurements indicates that 
southerly advection does increase longwave radiation, 
though not as frequently as katabatic winds 
enhance sensible and solar heating 
during more common melt events on sub-seasonal timescales, 
nor as strongly as net shortwave radiation on seasonal timescales.
%- ww: strongly or strong

Our study focuses on the ablation zone 
with its characteristically steep slopes that 
give rise to frequent katabatic winds. 
We doubt that katabatic winds 
dominate surface melt in the flatter interior. 
Model simulations \cite{Vihma2011} 
show that a slope of $5 \, m/km$, 
which is large for the accumulation zone \cite{Helm2014}, 
requires an elevation of at least $1500\,m$ 
for temperature downslope to increase during katabatic winds. 
Surface melt in the accumulation zone is, therefore, 
more susceptible to radiative energy fluxes, 
including longwave heating from liquid-containing clouds
\cite{Shupe2013, Miller2015}. 
However, surface melt in the ablation zone 
contributes the majority of surface mass loss, 
which is dominated by katabatic winds. 
As glaciers in Greenland retreat, so do the steeper slopes, 
which will always place this wind-dominated surface melt mechanism 
at the frontline of surface ablation.

%%

%  Numbered lines in equations:
%  To add line numbers to lines in equations,
%  \begin{linenomath*}
%  \begin{equation}
%  \end{equation}
%  \end{linenomath*}



%% Enter Figures and Tables near as possible to where they are first mentioned:
%
% DO NOT USE \psfrag or \subfigure commands.
%
% Figure captions go below the figure.
% Table titles go above tables;  other caption information
%  should be placed in last line of the table, using
% \multicolumn2l{$^a$ This is a table note.}
%
%----------------
% EXAMPLE FIGURES
%
% \begin{figure}
% \includegraphics{example.png}
% \caption{caption}
% \end{figure}
%
% Giving latex a width will help it to scale the figure properly. A simple trick is to use \textwidth. Try this if large figures run off the side of the page.
% \begin{figure}
% \noindent\includegraphics[width=\textwidth]{anothersample.png}
%\caption{caption}
%\label{pngfiguresample}
%\end{figure}
%
%
% If you get an error about an unknown bounding box, try specifying the width and height of the figure with the natwidth and natheight options. This is common when trying to add a PDF figure without pdflatex.
% \begin{figure}
% \noindent\includegraphics[natwidth=800px,natheight=600px]{samplefigure.pdf}
%\caption{caption}
%\label{pdffiguresample}
%\end{figure}
%
%
% PDFLatex does not seem to be able to process EPS figures. You may want to try the epstopdf package.
%

%
% ---------------
% EXAMPLE TABLE
%
% \begin{table}
% \caption{Time of the Transition Between Phase 1 and Phase 2$^{a}$}
% \centering
% \begin{tabular}{l c}
% \hline
%  Run  & Time (min)  \\
% \hline
%   $l1$  & 260   \\
%   $l2$  & 300   \\
%   $l3$  & 340   \\
%   $h1$  & 270   \\
%   $h2$  & 250   \\
%   $h3$  & 380   \\
%   $r1$  & 370   \\
%   $r2$  & 390   \\
% \hline
% \multicolumn{2}{l}{$^{a}$Footnote text here.}
% \end{tabular}
% \end{table}

%% SIDEWAYS FIGURE and TABLE
% AGU prefers the use of {sidewaystable} over {landscapetable} as it causes fewer problems.
%
% \begin{sidewaysfigure}
% \includegraphics[width=20pc]{figsamp}
% \caption{caption here}
% \label{newfig}
% \end{sidewaysfigure}
%
%  \begin{sidewaystable}
%  \caption{Caption here}
% \label{tab:signif_gap_clos}
%  \begin{tabular}{ccc}
% one&two&three\\
% four&five&six
%  \end{tabular}
%  \end{sidewaystable}

%% If using numbered lines, please surround equations with \begin{linenomath*}...\end{linenomath*}
%\begin{linenomath*}
%\begin{equation}
%y|{f} \sim g(m, \sigma),
%\end{equation}
%\end{linenomath*}

%%% End of body of article

%%%%%%%%%%%%%%%%%%%%%%%%%%%%%%%%
%% Optional Appendix goes here
%
% The \appendix command resets counters and redefines section heads
%
% After typing \appendix
%
%\section{Here Is Appendix Title}
% will show
% A: Here Is Appendix Title
%
%\appendix
%\section{Here is a sample appendix}

%%%%%%%%%%%%%%%%%%%%%%%%%%%%%%%%%%%%%%%%%%%%%%%%%%%%%%%%%%%%%%%%
%
% Optional Glossary, Notation or Acronym section goes here:
%
%%%%%%%%%%%%%%
% Glossary is only allowed in Reviews of Geophysics
%  \begin{glossary}
%  \term{Term}
%   Term Definition here
%  \term{Term}
%   Term Definition here
%  \term{Term}
%   Term Definition here
%  \end{glossary}

%
%%%%%%%%%%%%%%
% Acronyms
  \begin{acronyms}
  \acro{Acronym}
  Definition here
  \acro{EMOS}
  Ensemble model output statistics
  \acro{ECMWF}
  Centre for Medium-Range Weather Forecasts
  \end{acronyms}

%
%%%%%%%%%%%%%%
% Notation
%   \begin{notation}
%   \notation{$a+b$} Notation Definition here
%   \notation{$e=mc^2$}
%   Equation in German-born physicist Albert Einstein's theory of special
%  relativity that showed that the increased relativistic mass ($m$) of a
%  body comes from the energy of motion of the body—that is, its kinetic
%  energy ($E$)—divided by the speed of light squared ($c^2$).
%   \end{notation}




%%%%%%%%%%%%%%%%%%%%%%%%%%%%%%%%%%%%%%%%%%%%%%%%%%%%%%%%%%%%%%%%
%
%  ACKNOWLEDGMENTS
%
% The acknowledgments must list:
%
% >>>>	A statement that indicates to the reader where the data
% 	supporting the conclusions can be obtained (for example, in the
% 	references, tables, supporting information, and other databases).
%
% 	All funding sources related to this work from all authors
%
% 	Any real or perceived financial conflicts of interests for any
%	author
%
% 	Other affiliations for any author that may be perceived as
% 	having a conflict of interest with respect to the results of this
% 	paper.
%
%
% It is also the appropriate place to thank colleagues and other contributors.
% AGU does not normally allow dedications.


\acknowledgments
We thank the Geological Survey of Denmark and Greenland (GEUS) 
for providing data collected 
by the Programme for Monitoring of the Greenland Ice Sheet (PROMICE) 
and the Greenland Analogue Project (GAP). 
AWS data are downloaded from 
%- ww: better wording
the Programme for Monitoring of the Greenland Ice Sheet (PROMICE) \cite{Fausto2019}: \url{http://promice.org/PromiceDataPortal/api/download/AWS}.
Project supported by NASA AIST 80NSSC17K0540 and DOE E3SM DE-SC0019278.

% W.W. led the data analysis with input from all authors. D.V. and R.S.F. provided expertise in interpreting AWS data. M.K.L. provided expertise in interpreting wind fields. D.V. estimated the sensible and latent heat fluxed. R.S.F. estimated the annual ice ablation shown in Fig. 1a. W.W. wrote the manuscript with substantial input from C.S.Z. and D.V. All authors read and commented on the manuscript.


%% ------------------------------------------------------------------------ %%
%% References and Citations

%%%%%%%%%%%%%%%%%%%%%%%%%%%%%%%%%%%%%%%%%%%%%%%
%
% \bibliography{<name of your .bib file>} don't specify the file extension
%
% don't specify bibliographystyle
%%%%%%%%%%%%%%%%%%%%%%%%%%%%%%%%%%%%%%%%%%%%%%%

\bibliography{library}



%Reference citation instructions and examples:
%
% Please use ONLY \cite and \citeA for reference citations.
% \cite for parenthetical references
% ...as shown in recent studies (Simpson et al., 2019)
% \citeA for in-text citations
% ...Simpson et al. (2019) have shown...
%
%
%...as shown by \citeA{jskilby}.
%...as shown by \citeA{lewin76}, \citeA{carson86}, \citeA{bartoldy02}, and \citeA{rinaldi03}.
%...has been shown \cite{jskilbye}.
%...has been shown \cite{lewin76,carson86,bartoldy02,rinaldi03}.
%... \cite <i.e.>[]{lewin76,carson86,bartoldy02,rinaldi03}.
%...has been shown by \cite <e.g.,>[and others]{lewin76}.
%
% apacite uses < > for prenotes and [ ] for postnotes
% DO NOT use other cite commands (e.g., \citet, \citep, \citeyear, \nocite, \citealp, etc.).
%



\end{document}



More Information and Advice:

%% ------------------------------------------------------------------------ %%
%
%  SECTION HEADS
%
%% ------------------------------------------------------------------------ %%

% Capitalize the first letter of each word (except for
% prepositions, conjunctions, and articles that are
% three or fewer letters).

% AGU follows standard outline style; therefore, there cannot be a section 1 without
% a section 2, or a section 2.3.1 without a section 2.3.2.
% Please make sure your section numbers are balanced.
% ---------------
% Level 1 head
%
% Use the \section{} command to identify level 1 heads;
% type the appropriate head wording between the curly
% brackets, as shown below.
%
%An example:
%\section{Level 1 Head: Introduction}
%
% ---------------
% Level 2 head
%
% Use the \subsection{} command to identify level 2 heads.
%An example:
%\subsection{Level 2 Head}
%
% ---------------
% Level 3 head
%
% Use the \subsubsection{} command to identify level 3 heads
%An example:
%\subsubsection{Level 3 Head}
%
%---------------
% Level 4 head
%
% Use the \subsubsubsection{} command to identify level 3 heads
% An example:
%\subsubsubsection{Level 4 Head} An example.
%
%% ------------------------------------------------------------------------ %%
%
%  IN-TEXT LISTS
%
%% ------------------------------------------------------------------------ %%
%
% Do not use bulleted lists; enumerated lists are okay.
% \begin{enumerate}
% \item
% \item
% \item
% \end{enumerate}
%
%% ------------------------------------------------------------------------ %%
%
%  EQUATIONS
%
%% ------------------------------------------------------------------------ %%

% Single-line equations are centered.
% Equation arrays will appear left-aligned.

Math coded inside display math mode \[ ...\]
 will not be numbered, e.g.,:
 \[ x^2=y^2 + z^2\]

 Math coded inside \begin{equation} and \end{equation} will
 be automatically numbered, e.g.,:
 \begin{equation}
 x^2=y^2 + z^2
 \end{equation}


% To create multiline equations, use the
% \begin{eqnarray} and \end{eqnarray} environment
% as demonstrated below.
\begin{eqnarray}
  x_{1} & = & (x - x_{0}) \cos \Theta \nonumber \\
        && + (y - y_{0}) \sin \Theta  \nonumber \\
  y_{1} & = & -(x - x_{0}) \sin \Theta \nonumber \\
        && + (y - y_{0}) \cos \Theta.
\end{eqnarray}

%If you don't want an equation number, use the star form:
%\begin{eqnarray*}...\end{eqnarray*}

% Break each line at a sign of operation
% (+, -, etc.) if possible, with the sign of operation
% on the new line.

% Indent second and subsequent lines to align with
% the first character following the equal sign on the
% first line.

% Use an \hspace{} command to insert horizontal space
% into your equation if necessary. Place an appropriate
% unit of measure between the curly braces, e.g.
% \hspace{1in}; you may have to experiment to achieve
% the correct amount of space.


%% ------------------------------------------------------------------------ %%
%
%  EQUATION NUMBERING: COUNTER
%
%% ------------------------------------------------------------------------ %%

% You may change equation numbering by resetting
% the equation counter or by explicitly numbering
% an equation.

% To explicitly number an equation, type \eqnum{}
% (with the desired number between the brackets)
% after the \begin{equation} or \begin{eqnarray}
% command.  The \eqnum{} command will affect only
% the equation it appears with; LaTeX will number
% any equations appearing later in the manuscript
% according to the equation counter.
%

% If you have a multiline equation that needs only
% one equation number, use a \nonumber command in
% front of the double backslashes (\\) as shown in
% the multiline equation above.

% If you are using line numbers, remember to surround
% equations with \begin{linenomath*}...\end{linenomath*}

%  To add line numbers to lines in equations:
%  \begin{linenomath*}
%  \begin{equation}
%  \end{equation}
%  \end{linenomath*}



